\chapterimage{images/buildingyourapp/babysteps}

\chapter{Building your first app}

\section{Say goodbye to findViewById }

Kotlin Android Extensions are a Kotlin plugin that is included in the regular one, and that will allow finding views from \lstinline|Activities|, \lstinline|Fragments|, and \lstinline|Views| in a seamless way, without using the \lstinline|findViewById| method.

This plugin will generate some extra code that will allow you to access views in the layout XML, just as if they were properties with the name of the id you used in the layout definition.

It also builds a local view cache. So the first time a property is used, it will do a regular \lstinline|findViewById.| But next time, the view will be recovered from the cache, so the access will be faster.

\subsection{Plugin configuration}

%TODO show gradle file

\subsection{Using the plugin}

%TODO show example code of using the plugin from the DIce Roller

\section{Logger}
In the lesson, you added a \lstinline|ClickListener| to the roll button which generated a Toast on screen. This is mostly because we want to test the button functionality and see that it works. Whenever you find yourself in such a case, where you want to see whether some functionality is working, or just want to log some debug information, you can use the \lstinline|Log| functionality in Android. 

This section will not describe how to use this logging functionality, because it is well documented at \url{https://developer.android.com/studio/debug/am-logcat}. 

Make sure to apply the best practice, which is to remove all logging functionality as soon as the module or feature is fully implemented and thoroughly tested, before deployment to production. To enable your logs statements calls only during development phase, Android offers you the \lstinline|BuildConfig.DEBUG| property. This flag is set  automatically to \lstinline|false| when an application is deployed into an \gls{apk} for production and then, it’s set back to true during development.

\subsection{TIMBER}
There is a simple library called “Timber” which can log your messages and gives you the control over the flow of logs. Timber is a  library developed by  developer Jake Wharton, who has made a lot of interesting and helfpfull libraries.

We refer to \url{https://github.com/JakeWharton/timber} to see on how to use this library. 

In the example code of this lessons provided by us, you can see that an extra class has been created, which extends \lstinline|Application|. This  is a base class for maintaining global application state. You can provide your own implementation by creating a subclass and specifying the fully-qualified name of this subclass as the \lstinline|"android:name|" attribute in your Manifest's \lstinline|<application>| tag. The \lstinline|Application| class, or your subclass of the \lstinline|Application| class, is instantiated before any other class when the process for your application/package is created. 

It is there where we instantiate the Timber functionality, so we can use it throughout the application. 

\section{SemVer}
We want to make distinctions between the different versions of our application. Semantic Versioning (referred to, for short, as \gls{semver}), is a versioning system that has been on the rise over the last few years. With new plugins, addons, extensions, and libraries being built every day, having a universal way of versioning software development projects is a good thing to help us keep track of what’s going on.

A nice way to accomplish this is described in \url{https://medium.com/@maxirosson/versioning-android-apps-d6ec171cfd82}.

\section{Testing}

Testing your app is an integral part of the app development process. By running tests against your app consistently, you can verify your app's correctness, functional behavior, and usability before you release it publicly.

TODO: explain configuration

\subsection{Espresso}
TODO: explain espressoand adjustments


\subsection{JUnit}
TODO: explain Junit 




